\apendice{Documentación de usuario}

\section{Introducción}\label{introduccion-usuario}

En este manual se detallan los requerimientos de la aplicación, cómo
instalarla en un dispositivo Android e indicaciones sobre cómo
utilizarla correctamente. Todos los procedimientos aquí descritos se
encuentran también disponibles en formato video.

\section{Requisitos de usuarios}\label{requisitos-de-usuarios}

Los requisitos mínimos para poder hacer uso de la aplicación son:

\begin{itemize}
\tightlist
\item
  Contar con un dispositivo que posea Android 4.4 (\emph{KitKat} -- API
  19) o superior.
\item
  Para utilizar la característica de monitorización de la actividad, es
  necesario tener instalada la aplicación
  \href{https://play.google.com/store/apps/details?id=org.opencv.engine}{OpenCV
  Manager}.
\item
  También se necesita contar con permiso para acceder a la cámara del
  dispositivo.
\item
  Si se desea localizar los colmenares mediante GPS, es necesario contar
  con un dispositivo que lo soporte y conceder el permiso de
  localización a la aplicación.
\item
  Para acceder a la información meteorológica se requiere conexión a
  internet.
\end{itemize}

\section{Instalación}\label{instalacion}

La instalación se puede realizar de dos maneras: a través de Google Play
o instalando directamente el ejecutable de la aplicación en nuestro
dispositivo.

\subsection{Desde Google Play}\label{desde-google-play}

Google Play es una plataforma de distribución digital de aplicaciones
móviles para los dispositivos Android. GoBees se distribuye por esta
plataforma desde su versión 1.0.

\imagen{gobees-google-play}{GoBees en Google Play.}

Video-tutorial:
\url{http://gobees.io/help/videos/instalacion-google-play}

Para instalar la aplicación debemos realizar los siguientes pasos:

\begin{enumerate}
\def\labelenumi{\arabic{enumi}.}
\tightlist
\item
  Acceder a la aplicación Google Play.
\item
  Buscar el término ``GoBees''.
\item
  Entrar en la sección correspondiente a la aplicación.
\item
  Pulsar el botón instalar.
\item
  Cuando la instalación haya finalizado, pulsar sobre el botón abrir.
\item
  La instalación habrá finalizado y la aplicación estará lista para su
  uso.
\end{enumerate}

\imagenAncho{gobees-google-play-install}{Instalación desde Google Play}{0.5}

\subsection{Desde fichero ejecutable}\label{desde-fichero-ejecutable}

La otra opción, es realizar la instalación directamente desde el fichero
ejecutable de la aplicación. Estos ficheros poseen la extensión
\texttt{.apk}. Podemos conseguir la última versión del \texttt{.apk} de
GoBees desde \citep{github:gobees_apk}.

Video-tutorial: \url{http://gobees.io/help/videos/instalacion-apk}

Una vez descargado, tenemos que seguir los siguientes pasos:

\begin{enumerate}
\def\labelenumi{\arabic{enumi}.}
\tightlist
\item
  En primer lugar, hay que permitir la instalación de ``aplicaciones con
  orígenes desconocidos''. Para ello:

  \begin{enumerate}
  \def\labelenumii{\alph{enumii}.}
  \tightlist
  \item
    Ir a ajustes del dispositivo.
  \item
    Seguridad (o Privacidad).
  \item
    Activar ``Orígenes desconocidos''.
  \end{enumerate}
\item
  Ejecutar el fichero descargado.
\item
  Pulsar el botón instalar.
\item
  Cuando la instalación haya finalizado, pulsar sobre el botón abrir.
\item
  La instalación habrá finalizado y la aplicación estará lista para su
  uso.
\end{enumerate}

\section{Manual de usuario}\label{manual-de-usuario-1}

En esta sección se describe el uso de las diferentes funcionalidades de
la aplicación.

\subsection{Generar datos de muestra}\label{generar-datos-de-muestra}

Una de las mejores maneras de aprender a utilizar una aplicación es
indagando en ella. GoBees permite generar un colmenar de prueba, de tal
manera, que podemos explorar las diferentes secciones con datos reales.

Video-tutorial:
\url{http://gobees.io/help/videos/generar-colmenar-prueba}

Para generar los datos de prueba:

\begin{enumerate}
\def\labelenumi{\arabic{enumi}.}
\tightlist
\item
  Pulsar el botón menú.
\item
  Entrar en la sección ``Ajustes''.
\item
  Seleccionar la opción ``Generar datos de muestra''.
\item
  Se generará un colmenar con tres colmenas y tres grabaciones por
  colmena.
\end{enumerate}

\imagenAncho{sample-apiary}{Colmenar de muestra.}{0.5}

\subsection{Añadir un colmenar}\label{auxf1adir-un-colmenar}

Un colmenar hace referencia al lugar o recinto donde se poseen un
conjunto de colmenas. Un colmenar posee un nombre, una localización y
unas notas.

Video-tutorial: \url{http://gobees.io/help/videos/anadir-colmenar}

Para añadir un nuevo colmenar:

\begin{enumerate}
\def\labelenumi{\arabic{enumi}.}
\tightlist
\item
  Desde la pantalla principal.
\item
  Pulsar el botón ``+''.
\item
  Definir el nombre del colmenar (obligatorio).
\item
  Definir la localización del colmenar (opcional).

  \begin{enumerate}
  \def\labelenumii{\alph{enumii}.}
  \tightlist
  \item
    Se pueden introducir manualmente las coordenadas, indicando la
    latitud y la longitud en el sistema de coordenadas geográficas.
  \item
    Alternativamente, se puede obtener la localización actual
    automáticamente pulsando el botón situado en la parte derecha (se
    necesitan permisos de localización para utilizar esta
    característica).
  \end{enumerate}
\item
  Definir unas notas sobre el colmenar (opcional). En las notas se puede
  apuntar cualquier cosa relacionada con el colmenar en general.
\item
  Pulsar el botón {$\checkmark$} para guardar el nuevo colmenar.
\end{enumerate}

\imagenAncho{add-apiary}{Añadir colmenar.}{0.5}

\subsection{Editar un colmenar}\label{editar-un-colmenar}

Los detalles de un colmenar se pueden editar en cualquier momento.

Video-tutorial: \url{http://gobees.io/help/videos/editar-colmenar}

Para editar un colmenar existente:

\begin{enumerate}
\def\labelenumi{\arabic{enumi}.}
\tightlist
\item
  Desde la pantalla principal.
\item
  Pulsar el botón de menú asociado al colmenar a editar (tres puntos
  verticales situados en la esquina superior derecha).
\item
  Seleccionar la opción de editar.
\item
  Se abrirá la pantalla de edición, donde se podrán modificar los datos
  que se deseen.
\item
  Pulsar el botón {$\checkmark$} para actualizar los datos editados.
\end{enumerate}

\subsection{Eliminar un colmenar}\label{eliminar-un-colmenar}

Al eliminar un colmenar, se eliminan también todos los datos asociados a
este (información del colmenar, colmenas, grabaciones e información
meteorológica).

Video-tutorial: \url{http://gobees.io/help/videos/eliminar-colmenar}

Para eliminar un colmenar existente:

\begin{enumerate}
\def\labelenumi{\arabic{enumi}.}
\tightlist
\item
  Desde la pantalla principal.
\item
  Pulsar el botón de menú asociado al colmenar a eliminar (tres puntos
  verticales situados en la esquina superior derecha).
\item
  Seleccionar la opción de eliminar.
\item
  El colmenar se eliminará junto con toda su información.
\end{enumerate}

\subsection{Consultar la información meteorológica de un
colmenar}\label{consultar-la-informaciuxf3n-meteoroluxf3gica-de-un-colmenar}

Para poder consultar la información meteorológica de un colmenar se
necesita que este posea una localización y que el dispositivo esté
conectado a internet. Si se cumplen estos dos requisitos, la información
meteorológica del colmenar se actualizará automáticamente de forma
periódica.

Video-tutorial:
\url{http://gobees.io/help/videos/consultar-info-meteo-colmenar}

Para consultar la información meteorológica:

\begin{enumerate}
\def\labelenumi{\arabic{enumi}.}
\tightlist
\item
  Asegurarse de que el colmenar tiene definida una localización y que se
  posee conexión a internet.
\item
  En la lista de colmenares, se puede visualizar un resumen con la
  temperatura y situación meteorológica en cada colmenar.
\item
  Si se desea consultar la información en detalle, entrar en el colmenar
  a consultar.
\item
  Desplazarse a la pestaña ``info''.
\item
  En la parte inferior podremos visualizar todos los detalles de la
  situación meteorológica actual en ese colmenar.
\end{enumerate}

Se pueden cambiar las unidades meteorológicas, para ello:

\begin{enumerate}
\def\labelenumi{\arabic{enumi}.}
\tightlist
\item
  En la pantalla principal.
\item
  Pulsar el botón menú.
\item
  Entrar en la sección ``Ajustes''.
\item
  Seleccionar ``Unidades meteorológicas''.

  \begin{enumerate}
  \def\labelenumii{\alph{enumii}.}
  \tightlist
  \item
    Sistema métrico: ºC y km/h.
  \item
    Sistema imperial: ºF y mph.
  \end{enumerate}
\end{enumerate}

\imagenAncho{meteo-info}{Información meteorológica.}{0.5
}

\subsection{Visualizar un colmenar en el
mapa}\label{visualizar-un-colmenar-en-el-mapa}

GoBees nos permite visualizar fácilmente un determinado colmenar en un
mapa utilizando nuestra aplicación de mapas favorita. De esta manera,
podemos navegar hacia él o consultar cualquier detalle cartográfico.

Video-tutorial: \url{http://gobees.io/help/videos/ver-colmenar-mapa}

Para visualizar un colmenar en el mapa:

\begin{enumerate}
\def\labelenumi{\arabic{enumi}.}
\tightlist
\item
  Entrar en el colmenar a visualizar.
\item
  Desplazarse a la pestaña ``info''.
\item
  Pulsar el botón ``mapa'' situado a la derecha de la localización del
  colmenar.
\item
  Seleccionar la aplicación con la que se desea visualizar el colmenar.
\end{enumerate}

\subsection{Añadir una colmena}\label{auxf1adir-una-colmena}

Cada colmena pertenece a un colmenar y tiene un nombre y unas notas.
Además, se puede monitorizar su actividad de vuelo, dando lugar a
grabaciones.

Video-tutorial: \url{http://gobees.io/help/videos/anadir-colmena}

Para añadir una colmena en un determinado colmenar:

\begin{enumerate}
\def\labelenumi{\arabic{enumi}.}
\tightlist
\item
  Entrar en el colmenar al que pertenecerá.
\item
  Definir el nombre de la colmena (obligatorio).
\item
  Definir unas notas sobre la colmena (opcional). En las notas se puede
  apuntar cualquier cosa relacionada con la colmena en concreto.
\item
  Pulsar el botón {$\checkmark$} para guardar la nueva colmena.
\end{enumerate}

\subsection{Editar una colmena}\label{editar-una-colmena}

Los detalles de una colmena se pueden editar en cualquier momento.

Video-tutorial: \url{http://gobees.io/help/videos/editar-colmena}

Para editar una colmena existente:

\begin{enumerate}
\def\labelenumi{\arabic{enumi}.}
\tightlist
\item
  Entrar en el colmenar al que pertenece la colmena.
\item
  Pulsar el botón de menú asociado a la colmena a editar (tres puntos
  verticales situados en la esquina superior derecha).
\item
  Seleccionar la opción de editar.
\item
  Se abrirá la pantalla de edición, donde se podrán modificar los datos
  que se deseen.
\item
  Pulsar el botón {$\checkmark$} para actualizar los datos editados.
\end{enumerate}

\subsection{Eliminar una colmena}\label{eliminar-una-colmena}

Al eliminar una colmena, se eliminan también todos los datos asociados a
esta (información de la colmena y sus grabaciones).

Video-tutorial: \url{http://gobees.io/help/videos/eliminar-colmena}

Para eliminar una colmena existente:

\begin{enumerate}
\def\labelenumi{\arabic{enumi}.}
\tightlist
\item
  Entrar en el colmenar al que pertenece la colmena.
\item
  Pulsar el botón de menú asociado a la colmena a editar (tres puntos
  verticales situados en la esquina superior derecha).
\item
  Seleccionar la opción de eliminar.
\item
  La colmena se eliminará junto con toda su información.
\end{enumerate}

\subsection{Monitorizar la actividad de vuelo de una
colmena}\label{monitorizar-la-actividad-de-vuelo-de-una-colmena}

La actividad de vuelo, junto con información previa de la colmena y
conocimiento de las condiciones locales, permite conocer al apicultor el
estado de la colmena con bastante seguridad, pudiendo determinar si esta
necesita o no una intervención.

GoBees permite monitorizar este parámetro utilizando la cámara del
\emph{smartphone}.

Video-tutorial:
\url{http://gobees.io/help/videos/monitorizacion-act-vuelo}

Para monitorizar la actividad de vuelo es necesario colocar el
\emph{smartphone} de forma fija en posición cenital a la colmena. Para
esto, se puede utilizar un trípode o un soporte similar. En la siguiente
imagen se puede ver un ejemplo de colocación:

\imagenAncho{cenital}{Colocación del \emph{smartphone} en la colmena.}{0.75}

Para mejorar los resultados de la monitorización, es recomendable que el
suelo sea de un color claro y uniforme. Si posee maleza, se puede
colocar un cartón o similar, como se muestra en la imagen.

Una vez realizado en montaje, hay que seguir los siguientes pasos dentro
de la aplicación:

\begin{enumerate}
\def\labelenumi{\arabic{enumi}.}
\tightlist
\item
  Entrar en el colmenar al que pertenece la colmena a monitorizar.
\item
  Entrar en la colmena.
\item
  Pulsar en el botón de ``monitorización'' (situado en la parte inferior
  derecha con un icono de una cámara).
\item
  Se abrirá una ventana que permite previsualizar la monitorización.
\item
  Para configurar los parámetros de la monitorización, pulsar el botón
  ``ajustes'' (situado en la parte superior derecha). Se abrirá una
  pantalla con los siguientes ajustes:

  \begin{itemize}
  \tightlist
  \item
    \textbf{Mostrar salida del algoritmo}: si no se encuentra activado
    se previsualiza la imagen proveniente de la cámara. Si se activa, se
    muestran en verde las abejas detectadas y en rojo otros objetos en
    movimiento que el algoritmo no considera abejas. Además, en la
    esquina inferior derecha se puede visualizar el número total de
    abejas contadas en cada fotograma.
  \item
    \textbf{Modificar el tamaño de las regiones}: dependiendo de la
    distancia a la que esté situada la cámara, es posible que las abejas
    se visualicen demasiado pequeñas o demasiado grandes. Con esta
    opción, se puede agrandar o disminuir su silueta.
  \item
    \textbf{Min. área abeja}: la detección de una abeja se realiza por
    área. Si el contorno en movimiento detectado posee un área dentro de
    unos límites se considera una abeja. Este parámetro configura la
    cota inferior del área. Bien ajustado, permite descartar moscas y
    mosquitos.
  \item
    \textbf{Max. área abeja}: configura la cota superior del área.
    Permite descartar la mayoría de animales que pueden habitar en el
    colmenar (avispones, roedores, lagartos o cualquier animal de mayor
    tamaño).
  \item
    \textbf{Zoom}: permite configurar el zoom de la cámara para
    encuadrar la superficie deseada.
  \item
    \textbf{Frecuencia de muestreo}: determina el intervalo de tiempo
    entre un fotograma analizado y el siguiente a analizar. Es decir, si
    se establece en 1 segundo, la aplicación captará y analizará un
    fotograma cada segundo. Cuanto mayor sea el intervalo menor será el
    consumo de batería.
  \end{itemize}
\item
  Una vez configurados los parámetros correctamente, se puede iniciar la
  monitorización pulsado el botón blanco.
\item
  Se iniciará una cuenta atrás y comenzará la monitorización. Durante
  esta, la pantalla puede estar apagada para ahorrar batería. Se puede
  aprovechar la cuenta atrás para apagarla sin influir en la
  monitorización (al manipular el móvil siempre se producen
  trepidaciones).
\item
  Cuando se desee detener la monitorización, se debe pulsar el botón
  cuadrado rojo. Una vez pulsado, se guardará la grabación y se podrá
  acceder a los detalles de esta.
\end{enumerate}

\textbf{Nota:} Si se posee alguna aplicación de ahorro de batería es imprescindible
añadir una excepción a la aplicación GoBees para que esta se pueda
ejecutar en segundo plano sin restricciones. Si no, la aplicación puede
ser cerrada durante la monitorización.

\imagen{monitoring-settings}{Ajustes de monitorización.}

\subsection{Ver los detalles de una
grabación}\label{ver-los-detalles-de-una-grabacion}

Al monitorizar una colmena se genera lo que denominamos una grabación.
Una grabación contiene los datos de actividad de vuelo de la colmena.

Video-tutorial: \url{http://gobees.io/help/videos/ver-grabacion}

Para ver los detalles de una grabación:

\begin{enumerate}
\def\labelenumi{\arabic{enumi}.}
\tightlist
\item
  Entrar en el colmenar al que pertenece la colmena monitorizada.
\item
  Entrar en la colmena.
\item
  Pulsar en la grabación sobre la que se está interesado.
\item
  Se mostrará una pantalla con dos gráficos.

  \begin{enumerate}
  \def\labelenumii{\alph{enumii}.}
  \tightlist
  \item
    El gráfico principal muestra la actividad de vuelo. En el eje de las
    Y se representa el número de abejas en vuelo y en las X los
    instantes de tiempo. Si se pulsa sobre un punto del gráfico, se
    obtiene la medida exacta en ese punto.
  \item
    El gráfico inferior muestra la información meteorológica. Existe un
    selector con tres botones: temperatura, precipitaciones y viento.
    Según se presione en uno u otro, se muestra su gráfico
    correspondiente.
  \end{enumerate}
\item
  Con ambos gráficos se puede interpretar la actividad de vuelo de la
  colmena y determinar si es una actividad normal o la colmena necesita
  una intervención.
\end{enumerate}

\imagenAncho{recording-detail}{Detalle de la grabación.}{0.5}

\subsection{Eliminar una grabación}\label{eliminar-una-grabaciuxf3n}

Al eliminar una grabación, se eliminan también todos los datos asociados
a esta.

Video-tutorial: \url{http://gobees.io/help/videos/eliminar-grabacion}

Para eliminar una grabación existente:

\begin{enumerate}
\def\labelenumi{\arabic{enumi}.}
\tightlist
\item
  Entrar en el colmenar al que pertenece la colmena monitorizada.
\item
  Entrar en la colmena.
\item
  Localizar la grabación y pulsar el botón de menú asociado a esta (tres
  puntos verticales situados en la esquina superior derecha).
\item
  Seleccionar la opción de eliminar.
\item
  La grabación se eliminará junto con toda su información.
\end{enumerate}

\subsection{Eliminar toda la información de la
aplicación}\label{eliminar-toda-la-informaciuxf3n-de-la-aplicaciuxf3n}

Si por algún motivo se desea resetear toda la información almacenada en
la aplicación, esta cuenta una opción para ello.

Video-tutorial: \url{http://gobees.io/help/videos/eliminar-datos}

Para eliminar toda la información de la aplicación:

\begin{enumerate}
\def\labelenumi{\arabic{enumi}.}
\tightlist
\item
  Pulsar el botón menú.
\item
  Entrar en la sección ``Ajustes''.
\item
  Seleccionar la opción ``Borrar todos los datos''.
\item
  Todos los datos de la aplicación serán borrados. La aplicación volverá
  al mismo estado que cuando se instaló.
\end{enumerate}

\subsection{Consultar la información sobre la
aplicación}\label{consultar-la-informaciuxf3n-sobre-la-aplicaciuxf3n}

Para conocer la versión instalada de la aplicación, los cambios
introducidos en las diferentes versiones, la licencia o el autor de esta
hay que acceder a la sección ``Acerca de GoBees''.

Video-tutorial: \url{http://gobees.io/help/videos/acerca-gobees}

Para acceder a la sección ``Acerca de GoBees'':

\begin{enumerate}
\def\labelenumi{\arabic{enumi}.}
\tightlist
\item
  Pulsar el botón menú.
\item
  Entrar en la sección ``Acerca de GoBees''.
\item
  En ella se puede visualizar la versión de la aplicación, el autor y
  las bibliotecas utilizadas para su desarrollo.
\item
  Si se presiona el botón ``Website'' se accede a la página web de
  GoBees.
\item
  Si se presiona el botón ``Licencia'' se visualiza una copia de la
  licencia de la aplicación.
\item
  Si se presiona el botón ``\emph{Changelog}'' se visualizan los cambios
  introducidos en cada versión.
\end{enumerate}

\imagenAncho{about-gobees}{Sobre GoBees.}{0.5}
