\capitulo{4}{Técnicas y herramientas}

\section{Metodologías}\label{metodologias}

\subsection{Scrum}\label{scrum}

Scrum es un marco de trabajo para el desarrollo de \emph{software} que se
engloba dentro de las metodologías ágiles. Aplica una estrategia de
trabajo iterativa e incremental a través de iteraciones (\emph{sprints})
y revisiones \citep{wiki:scrum}.

\subsection{\emph{Test-Driven Development} (TDD)}\label{test-driven-development-tdd}

Es una práctica de desarrollo de \emph{software} que se basa en la repetición
de un ciclo corto de desarrollo: transformar requerimientos a test,
desarrollar el código necesario para pasar los test y posteriormente
refactorizar el código. Esta práctica obliga a los desarrolladores a
analizar cuidadosamente las especificaciones antes de empezar a escribir
código, fomenta la escritura de test, la simplicidad del código y
aumenta la productividad. Como resultado se obtiene \emph{software} más seguro
y de mayor calidad \citep{wiki:tdd}.

\subsection{Gitflow}\label{gitflow}

Gitflow es un flujo de trabajo con Git que define un modelo estricto de ramas
diseñado en torno a los lanzamientos de proyecto. En la rama \emph{main}
se hospeda la última versión estable del proyecto. La rama
\emph{develop} contiene los últimos desarrollos realizados para el
siguiente lanzamiento. Por cada característica que se vaya a implementar
se crea una \emph{feature branch}. La preparación del siguiente
lanzamiento se realiza en una \emph{release branch}. Si aparece un fallo
en producción, este se soluciona en una \emph{hotfix branch} \citep{git:gitflow}.

\subsection{Técnica Pomodoro}\label{pomodoro}

La técnica Pomodoro es un método para incrementar la productividad aprovechando 
mejor el tiempo. Para aplicarla, se divide la tarea a realizar en intervalos de
25 minutos, llamados \emph{pomodoros}. Durante estos intervalos se debe evitar cualquier 
distracción que nos desvíe de la tarea. Después de cada Pomodoro se descansa 5 
minutos, menos en los múltiplos de cuatro que se realiza un descanso de 30 minutos
\citep{wiki:pomodoro}.

Se ha utilizado la aplicación \href{https://www.microsoft.com/es-es/store/p/focus-10/9nblggh5g2xh}{Focus 10} como temporizador.

\section{Patrones de diseño}\label{patrones-de-diseno}

\subsection{\emph{Model-View-Presenter} (MVP)}\label{model-view-presenter-mvp}

MVP es un patrón de arquitectura derivado del
\emph{Model--View--Controller} (MVC). Permite separar los datos internos
del modelo de una vista pasiva y enlazarlos mediante el \emph{presenter}
que maneja toda la lógica de la aplicación \citep{pattern:mvp}. 
Posee tres capas:

\begin{itemize}
\tightlist
\item
  \textit{\textbf{Model}}: almacena y proporciona los datos internos.
\item
  \textit{\textbf{View}}: maneja la visualización de los datos (del modelo).
  Propaga las acciones de usuario al \emph{presenter}.
\item
  \textit{\textbf{Presenter}}: enlaza las dos capas anteriores. Sincroniza los
  datos mostrados en la vista con los almacenados en el modelo y actúa
  ante los eventos de usuario propagados por la vista.
\end{itemize}

\imagen{mvp}{Patrón MVP.}

\subsection{Patrón repositorio}\label{patron-repositorio}

El patrón repositorio proporciona una abstracción de la implementación
del acceso a datos con el objetivo de que este sea transparente a la
lógica de negocio de la aplicación. Por ejemplo, las fuentes de datos
pueden ser una base de datos, un \emph{web service}, etc. El repositorio
media entre la capa de acceso a datos y la lógica de negocio de tal
forma que no existe ninguna dependencia entre ellas. Consiguiendo
desacoplar, mantener y testear más fácilmente el código y permitiendo la
reutilización del acceso a datos desde cualquier cliente \citep{pattern:repository}.

\imagen{repository_pattern}{Patrón Repositorio.}

\section{Control de versiones}\label{control-de-versiones}

\begin{itemize}
\tightlist
\item
  Herramientas consideradas: \href{https://git-scm.com/}{Git} y
  \href{https://subversion.apache.org/}{Subversion}.
\item
  Herramienta elegida: \href{https://git-scm.com/}{Git}.
\end{itemize}

Git es un sistema de control de versiones distribuido. A día de hoy, es
el sistema con mayor número de usuarios. La principal diferencia con
Subversion es su carácter descentralizado, que permite a cada
desarrollador tener una copia en local del repositorio completo. Git
se distribuye bajo la licencia de \emph{software} libre GNU LGPL v2.1.

\section{\emph{Hosting} del repositorio}\label{hosting-del-repositorio}

\begin{itemize}
\tightlist
\item
  Herramientas consideradas: \href{https://github.com/}{GitHub},
  \href{https://bitbucket.org/}{Bitbucket} y
  \href{https://gitlab.com/}{GitLab}.
\item
  Herramienta elegida: \href{https://github.com/}{GitHub}.
\end{itemize}

GitHub es la plataforma web de hospedaje de repositorios por excelencia.
Ofrece todas las funcionalidades de Git, revisión de código,
documentación, \emph{bug tracking}, gestión de tareas, \emph{wikis}, red
social\ldots{} y numerosas integraciones con otros servicios. Es
gratuita para proyectos \emph{open source}.

Utilizamos GitHub como plataforma principal donde hospedamos el código
del proyecto, la gestión de proyecto (gracias a ZenHub) y la
documentación. Además, el repositorio está integrado con varios
servicios de integración continua.

\section{Gestión del proyecto}\label{gestion-del-proyecto}

\begin{itemize}
\tightlist
\item
  Herramientas consideradas: \href{https://www.zenhub.com/}{ZenHub},
  \href{https://trello.com/}{Trello}, \href{https://waffle.io/}{Waffle},
  \href{https://www.versionone.com/}{VersionOne},
  \href{https://xp-dev.com/}{XP-Dev} y \href{https://github.com/}{GitHub
  Projects}.
\item
  Herramienta elegida: \href{https://www.zenhub.com/}{ZenHub}.
\end{itemize}

ZenHub es una herramienta de gestión de proyectos totalmente integrada
en GitHub. Proporciona un tablero canvas en donde cada tarea
representada se corresponde con un \emph{issue} nativo de GitHub. Cada
tarea se puede priorizar dependiendo de su posición en la lista, se le
puede asignar una estimación, uno o varios responsables y el
\emph{sprint} al que pertenece. ZenHub también permite visualizar el
gráfico \emph{burndown} de cada \emph{sprint}. Es gratuita para
proyectos pequeños (máx. 5 colaboradores) o proyectos \emph{open
source}.

\section{Comunicación}\label{comunicacion}

\begin{itemize}
\tightlist
\item
  Herramientas consideradas: email y
  \href{https://gobees.slack.com/}{Slack}.
\item
  Herramienta elegida: \href{https://gobees.slack.com/}{Slack}.
\end{itemize}

Slack es una herramienta de colaboración de equipos que ofrece salas de
chat, mensajes directos y llamadas VoIP. Posee un buscador que permite
encontrar todo el contenido generado dentro de Slack. Además, ofrece un
gran número de integraciones con otros servicios. En nuestro proyecto
vamos a utilizar la integración con GitHub para crear un canal que sirva
de \emph{log} de todas las acciones realizadas en GitHub. Slack ofrece una
versión gratuita que provee las características principales.

\section{Entorno de desarrollo integrado
(IDE)}\label{entorno-de-desarrollo-integrado-ide}

\subsection{Java}\label{java}

\begin{itemize}
\tightlist
\item
  Herramientas consideradas:
  \href{https://www.jetbrains.com/idea/}{IntelliJ IDEA} y
  \href{https://eclipse.org/}{Eclipse}.
\item
  Herramienta elegida: \href{https://www.jetbrains.com/idea/}{IntelliJ
  IDEA}.
\end{itemize}

IntelliJ IDEA es un IDE para Java desarrollado por JetBrains. Posee un
gran número de herramientas para facilitar el proceso de escritura,
revisión y refactorización del código. Además, permite la integración de
diferentes herramientas y posee un ecosistema de \emph{plugins} para
ampliar su funcionalidad. Su versión \emph{community} está disponible
bajo la licencia Apache 2. Aunque también es posible adquirir la versión
\emph{Ultimate} gratuitamente si se es estudiante.

\subsection{Android}\label{android}

\begin{itemize}
\tightlist
\item
  Herramientas consideradas:
  \href{https://developer.android.com/studio/index.html}{Android Studio}
  y \href{https://eclipse.org/}{Eclipse}.
\item
  Herramienta elegida:
  \href{https://developer.android.com/studio/index.html}{Android
  Studio}.
\end{itemize}

Android Studio es el IDE oficial para el desarrollo de aplicaciones
Android. Está basado en IntelliJ IDEA de JetBrains. Proporciona soporte
para Gradle, emulador, editor de \emph{layouts}, refactorizaciones
específicas de Android, herramientas Lint para detectar problemas de
rendimiento, uso, compatibilidad de versión, etc. Se distribuye bajo la
licencia Apache 2.

\subsection{Markdown}\label{markdown}

\begin{itemize}
\tightlist
\item
  Herramientas consideradas: \href{https://stackedit.io/}{StackEdit} y
  \href{http://pad.haroopress.com/}{Haroopad}.
\item
  Herramienta elegida: \href{http://pad.haroopress.com/}{Haroopad}.
\end{itemize}

Haroopad es un editor de documentos Markdown. Soporta Github Flavored
Markdown y Mathematics Expression, además de contar con un gran número
de extensiones. Se distribuye bajo licencia GNU GPL v3.0.

\subsection{LaTeX}\label{latex}

\begin{itemize}
\tightlist
\item
  Herramientas consideradas:
  \href{https://www.sharelatex.com/}{ShareLaTeX} y
  \href{http://www.xm1math.net/texmaker/}{Texmaker}.
\item
  Herramienta elegida:
  \href{http://www.xm1math.net/texmaker/}{Texmaker}.
\end{itemize}

Texmaker es un editor gratuito y multiplataforma para \LaTeX. Integra la
mayoría de herramientas necesarias para la escritura de documentos en
\LaTeX (PdfLaTeX , BibTeX, makeindex, etx). Además, incluye corrector
ortográfico, auto-completado, resaltado de sintaxis, visor de PDFs
integrado, etc. Está licenciado bajo GNU GPL v2.

\section{Documentación}\label{documentacion}

\begin{itemize}
\tightlist
\item
  Herramientas consideradas:
  \href{https://www.latex-project.org/}{LaTeX},
  \href{http://daringfireball.net/projects/markdown/}{Markdown},
  \href{http://docutils.sourceforge.net/docs/ref/rst/restructuredtext.html}{reStructuredText} y  
  \href{https://products.office.com/es-es/word}{Microsoft Word}
\item
  Herramienta elegida:
  \href{http://daringfireball.net/projects/markdown/}{Markdown} +
  \href{http://docutils.sourceforge.net/docs/ref/rst/restructuredtext.html}{reStructuredText} +
  \href{https://www.latex-project.org/}{\LaTeX}.
\end{itemize}

La documentación se ha desarrollado en Markdown y reStructuredText para integrarla con el
servicio de documentación continua \href{https://readthedocs.org/}{Read
the Docs}. Una vez terminada, se ha exportado a \LaTeX utilizando el
conversor \href{http://pandoc.org/}{Pandoc}.

Markdown es un lenguaje de marcado ligero en texto plano que puede ser
exportado a numerosos formatos como HTML o PDF. Su filosofía es que el
lenguaje de marcado sea fácil de escribir y leer. Markdown es
ampliamente utilizado para la escritura de archivos README, en foros
como StackOverflow o en herramientas de comunicación como Slack.

reStructuredText es uno de los leguajes de marcado ligero en los que se 
inspiró Markdown. Su principal aplicación es la escritura de documentación 
de Python junto con el sistema de generación de documentación Sphinx.

\LaTeX es un sistema de composición de textos que genera documentos con
una alta calidad tipográfica. Es ampliamente utilizado para la
generación de artículos y libros científicos, principalmente por su
potencia a la hora de representar expresiones matemáticas.

\section{Servicios de integración
continua}\label{servicios-de-integraciuxf3n-continua}

\subsection{Compilación y testeo}\label{compilacion-y-testeo}

\begin{itemize}
\tightlist
\item
  Herramientas consideradas: \href{https://travis-ci.org/}{TravisCI} y
  \href{https://circleci.com/}{CircleCI}.
\item
  Herramienta elegida: \href{https://travis-ci.org/}{TravisCI}.
\end{itemize}

Travis es una plataforma de integración continua en la nube para
proyectos alojados en GitHub. Permite realizar una \emph{build} del
proyecto y testearla automáticamente cada vez que se realiza un
\emph{commit}, devolviendo un informe con los resultados. Es gratuita
para proyectos \emph{open source}.

\subsection{Cobertura de código}\label{cobertura-de-codigo}

\begin{itemize}
\tightlist
\item
  Herramientas consideradas: \href{https://coveralls.io/}{Coveralls} y
  \href{https://codecov.io/}{Codecov}.
\item
  Herramienta elegida: \href{https://codecov.io/}{Codecov}.
\end{itemize}

Codecov es una herramienta que permite medir el porcentaje de código que
está cubierto por test unitarios. Además, realiza representaciones visuales de
la cobertura y gráficos de su evolución. Posee una extensión de
navegador para GitHub que permite visualizar por cada archivo de código
que líneas están cubiertas por un test y cuáles no. Es gratuita para
proyectos \emph{open source}.

\subsection{Calidad del código}\label{calidad-del-codigo}

\begin{itemize}
\tightlist
\item
  Herramientas consideradas:
  \href{https://codeclimate.com/}{Codeclimate},
  \href{https://sonarqube.com/}{SonarQube} y
  \href{https://www.codacy.com/}{Codacy}.
\item
  Herramientas elegidas: \href{https://codeclimate.com/}{Codeclimate} y
  \href{https://sonarqube.com/}{SonarQube}.
\end{itemize}

Codeclimate es una herramienta que realiza revisiones de código
automáticamente. Es gratuita para proyectos \emph{open source}. En
nuestro proyecto hemos activado los siguientes motores de chequeo:
\href{https://docs.codeclimate.com/docs/checkstyle}{checkstyle},
\href{https://docs.codeclimate.com/docs/fixme}{fixme},
\href{https://docs.codeclimate.com/docs/markdownlint}{markdownlint} y
\href{https://docs.codeclimate.com/docs/pmd}{pmd}.

SonarQube es una plataforma de código abierto para la revisión continua
de la calidad de código. Permite detectar código duplicado, violaciones
de estándares, cobertura de tests unitarios, \emph{bugs} potenciales,
etc.

\subsection{Revisión de dependencias}\label{revision-de-dependencias}

\href{https://www.versioneye.com/}{VersionEye} es una herramienta que monitoriza las dependencias del
proyecto y envía notificaciones cuando alguna de estas está
desactualizada, es vulnerable o viola la licencia del proyecto. Posee
una versión gratuita con ciertas limitaciones.

\subsection{Documentación continua}\label{documentacion-continua}

\href{https://readthedocs.org/}{Read the Docs} es un servicio de documentación continua que permite crear
y hospedar una página web generada a partir de los distintos ficheros
Markdown o reStructuredText de la documentación. Cada vez que se realiza
un \emph{commit} en el repositorio se actualiza la versión hospedada. La
página web posee un buscador, da soporte para diferentes versiones del
proyecto y soporta internacionalización. Además, permite exportar la
documentación en varios formatos (pdf, epub, html, etc.). El servicio es
totalmente gratuito, sostenido por donaciones y subscripciones
\emph{Gold}.

\section{Sistemas de construcción automática del
\emph{software}}\label{sistemas-de-construccion-automuxe1tica-del-software}

\subsection{Maven}\label{maven}

\href{https://maven.apache.org/}{Maven} es una herramienta para
automatizar el proceso de construcción del \emph{software} (compilación,
testeo, empaquetado, etc.) enfocada a proyectos Java. Básicamente
describe cómo se tiene que construir el \emph{software} y cuáles son sus
dependencias.

\subsection{Gradle}\label{gradle}

\href{https://gradle.org/}{Gradle} es una herramienta similar a Maven
pero basada en el lenguaje de programación orientado a objetos Groovy.
El sistema de construcción de Android Studio está basado en Gradle y es
actualmente el único soportado de forma oficial para Android.

\section{Librerías}\label{libreruxedas}

\subsection{\texorpdfstring{\emph{Android Support
Library}}{Android Support Library}}\label{android-support-library}

La
\href{https://developer.android.com/topic/libraries/support-library/}{librería
de soporte de Android} facilita algunas características que no se
incluyen en el \emph{framework} oficial. Proporciona compatibilidad a
versiones antiguas con las últimas características, incluye elementos
para la interfaz adicionales y utilidades extra.

\subsection{Espresso}\label{espresso}

\href{https://google.github.io/android-testing-support-library/docs/espresso/}{Espresso}
es un framework de \emph{testing} para Android incluido en la librería
de soporte para \emph{testing} en Android. Provee una API para escribir
UI test que simulen las interacciones de usuario con la app.

\subsection{Google Guava}\label{google-guava}

\href{https://github.com/google/guava}{Google Guava} agrupa un conjunto
de librerías comunes para Java. Proporciona utilidades básicas para
tareas cotidianas, una extensión del \emph{Java collections framework}
(JCF) y otras extensiones como programación funcional, almacenamiento en
caché, objetos de rango o \emph{hashing}.

\subsection{Google Play Services}\label{google-play-services}

\href{https://developers.google.com/android/guides/overview}{Google Play
Services} es una librería que permite a las aplicaciones de terceros
utilizar características de aplicaciones de Google como Maps, Google\texttt{+},
etc. En nuestro caso se ha hecho uso de su servicio de localización, que
utiliza varias fuentes de datos (GPS, red y \emph{wifi}) para ubicar el
dispositivo rápidamente.

\subsection{JavaFX}\label{javafx}

\href{http://docs.oracle.com/javase/8/javase-clienttechnologies.htm}{JavaFX}
es una librería para la creación de interfaces gráficas en Java.

\subsection{JUnit}\label{junit}

\href{http://junit.org/junit4/}{JUnit} es un \emph{framework} para Java
utilizado para realizar pruebas unitarias.

\subsection{Material Design}\label{material-design}

\href{https://material.io/guidelines/}{Material Design} es una guía de
estilos enfocada a la plataforma Android, pero aplicable a cualquier
otra plataforma. Fue presentada en el Google I/O 2014 y se adoptó en
Android a partir de la versión 5.0 (Lollipop). Se basa en objetos
materiales, piezas colocadas en un espacio (lugar) y con un tiempo
(movimiento) determinado.

\subsection{Mockito}\label{mockito}

\href{http://mockito.org/}{Mockito} es un \emph{framework} de
\emph{mocking} que permite crear objetos \emph{mock} fácilmente. Estos
objetos simulan parte del comportamiento de una clase. Mockito está
basado en EasyMock, mejorando su sintaxis haciendo los test más simples
y fáciles de leer y con mensajes de error descriptivos.

\subsection{MPAndroidChart}\label{mpandroidchart}

\href{https://github.com/PhilJay/MPAndroidChart}{MPAndroidChart} es una
librería para la creación de gráficos en Android.

\subsection{OpenCV}\label{opencv}

\href{www.opencv.org}{OpenCV} es un paquete \emph{Open Source} de visión
artificial que contiene más de 2500 librerías de procesamiento de
imágenes y visión artificial, escritas en C/C\texttt{++} a bajo/medio nivel. Se
distribuye gratuitamente bajo una licencia \emph{BSD} desde hace más de
una década. Posee una comunidad de más de 50.000 usuarios alrededor de
todo el mundo y se ha descargado más de 8 millones de veces.

Aunque OpenCV está escrito en C/C\texttt{++} posee \emph{wrappers} para varias
plataformas, entre ellas Android, en donde da soporte a las principales
arquitecturas de CPU. Desde hace unos años, también soporta CUDA para el
desarrollo en GPU tanto en escritorio como en móvil, aunque en esta
última el soporte es todavía reducido.

\subsection{OpenWeatherMaps}\label{openweathermaps}

\href{http://openweathermap.org/}{OpenWeatherMap} es un servicio online
que proporciona información meteorológica. Está inspirado en
OpenStreetMap y su filosofía de hacer accesible la información a la
gente de forma gratuita. Utiliza distintas fuentes de datos desde
estaciones meteorológicas oficiales, de aeropuertos, radares e incentiva
a los propietarios de estaciones meteorológicas a conectarlas a su red.
Proporciona una API que permite realizar hasta 60 llamadas por segundo
de forma gratuita.

\subsection{PowerMock}\label{powermock}

\href{https://github.com/powermock/powermock}{PowerMock} es una librería
de \emph{testing} que permite la creación de \emph{mocks} de métodos
estáticos, constructores, clases finales o métodos privados.

\subsection{Realm}\label{realm}

\href{https://realm.io/products/realm-mobile-database/}{Realm} es una
base de datos orientada a objetos enfocada a dispositivos móviles. Se
definen como la alternativa a SQLite y presumen de ser más rápidos que
cualquier ORM e incluso que SQLite puro. Posee una API muy intuitiva que
facilita en gran medida el acceso a datos.

\newpage
\section{Desarrollo web}\label{pagina-web}

\subsection{GitHub Pages}\label{github-pages}

\href{https://pages.github.com/}{GitHub Pages} es un servicio de hosting
estático que permite a proyectos que utilicen un repositorio de GitHub 
hospedar su página web en el propio repositorio. Permite utilizar Jekyll, un generador de sitios
estáticos. No soporta tecnologías del lado de servidor como PHP, Ruby,
Python, etc.

\subsection{Bootstrap}\label{bootstrap}

\href{http://getbootstrap.com/}{Bootstrap} es un \emph{framework} para
desarrollo \emph{front-end}. Contiene una serie de componentes ya
implementados que facilitan y agilizan el diseño. Está desarrollado
siguiendo la filosofía \emph{mobile first}.

\section{Otras herramientas}\label{otras-herramientas}

\subsection{Mendeley}\label{mendeley}

\href{https://www.mendeley.com/}{Mendeley} es un gestor de referencias
bibliográficas. Permite añadir referencias de varias formas, visualizar
los documentos, etiquetarlos, compartirlos, etc. Posteriormente se puede
exportar todo el catálogo a un fichero BibTex para ser utilizadas desde
\LaTeX.

\subsection{Creately}\label{creately}

\href{https://creately.com/}{Creately} es una aplicación web que permite
crear todo tipo de diagramas altamente personalizables. Aunque posee una
versión gratuita limitada, se optó por pagar un mes de subscripción al
valorar que realmente iba a ser de utilidad.
